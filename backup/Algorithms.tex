A cryptosystem consists of a plaintext space $\mathcal{P}$, a ciphertext space $\mathcal{C}$ and a key space $\mathcal{K}$. There is an encryption algorithm $Enc: \mathcal{K}$ x $\mathcal{P} \rightarrow \mathcal{C}$ and a decryption algorithm $Dec: \mathcal{K}$ x $\mathcal{C} \rightarrow \mathcal{P}$. For each $k \in \mathcal{K}$ and $p \in \mathcal{P}$, it is $Dec(Enc(p)_k)_k = p$. In a communication model as introduced by Shannon~\cite{shannon}, a cryptosystem plays an important role providing security to the information from an attacker, i.e., a malicious third entity. In the communication model, there is a sender and a receiver with a public communication channel. The sender aims at sending the plaintext (information) in a confidential way to the receiver. Confidentiality can be achieved with a cryptosystem and an additional secure channel of low bandwidth.  

Symmetric key cryptography assumes a secure channel that can not be eavesdropped by an attacker, and it is used by the sender to transmit a secret key $k$ to the receiver. Given $p$, $k$ and the cryptosystem, the sender can construct the ciphertext $c$ and send it to the receiver over the public communication channel. The receiver can then reconstruct the plaintext $p$, given $c$, $k$, and the cryptosystem. 
Symmetric key cryptography 
%can be up to 1000 times faster than asymmetric cryptography~\cite{mandal2012evaluation}, and this 
is relevant for wearable networks, in which devices and communication have constrained resources, and applications demand for low response time. %For symmetric key cryptography, parties share a common key. 
In this context, the attacker's main goal lies in recovering $p$ or $k$ and, according to Kerckhoff’s principle, an attacker knows the specification of the cryptosystem and has access to the ciphertext $c$. Hence, the security level in the communication model is exponential to the key size and depends on how strong is the cryptosystem. % when the attacker has full familiarity with its operation (Shannon's maxim and Kerchoff’s principle).

This work focuses on block and stream ciphers, i.e., two different classes of symmetric algorithms. A block cipher is a cryptosystem with $\mathcal{P} = \mathcal{C} =  \mathbb{F}^n$ for a block size $n$, where  $\mathbb{F}$ denotes the Galois field of two elements and $\mathbb{F}^n$ is the vector space of dimension $n$ over $\mathbb{F}$. For each key $k$, the encryption function $Enc(p)_k$ is a permutation. In the most general case, the $\mathcal{K}$ corresponds to the set of permutations of size $2^n!$, where a single $k$ is represented by a table of size $2^n$. It is reasonable to use a subset of the permutations, which can
be generated with a small key. To encrypt messages longer than the block size,
a mode of operation is employed. 

A stream cipher encrypts binary digits of a plaintext one at a time. %, using an encryption transformation which varies with time. 
In a nutshell, it consists of an internal state $x \in \mathcal{X}$, an update
function $L: \mathcal{X} \rightarrow \mathcal{X}$ and an output function $f: \mathcal{X} \rightarrow \mathcal{Z}$, where $\mathcal{Z}$ is called the keystream
alphabet. An output $z \in \mathcal{Z}$ at time $t$ is produced according to $zt = f(xt)$, where $xt = Lt(x)$
and $x$ is the initial state. The initial state $x$ is produced by an initialization function
from the secret key $k$ and an initialization vector. The stream of outputs $z_0, z_1, \ldots$ is called the keystream. Each output symbol is then combined with the
corresponding plaintext symbol to produce a ciphertext symbol.

%Figures~\ref{} generically illustrates block and stream ciphers, respectively. 
XTEA, XXTEA, SKIPJACK and RC2 are block ciphers; whereas RC4 is a stream cipher. %, handling one bit at a time. 
%The choice of 
These encryption algorithms have the % considered their 
capability of %, 
handling thresholds related to security, cost and performance, being well-known as ``light'' algorithms. In addition to these five lightweight algorithms, briefly described in the next paragraphs, we have initially considered others, e.g., KSEED, TWOFISH and CAST5. However, they have showed to be impractical for the current wearable device architecture due to the excessive memory use. % by data structures. % as matrices.

The eXtension to TEA (XTEA) and the Corrected Block TEA (XXTEA) encryption algorithms %is simple and relies on basic operations. It 
employ a 128-bit key and blocks of 64-bits. XTEA operates in 64 rounds and XXTEA relies in a variable number of rounds. In both, permutations follow simple operations, e.g., addition, shifting and XOR. For recovering a key, cryptanalysis estimates at least $2^{63.6}$ plaintexts and $2^{126.15}$ ciphertexts for XTEA, and $2^{59}$ plaintexts for XXTEA. %~\cite{moon2002impossible}.
%, using differential attacks. 
XTEA decryption process is vulnerable to attacks because the algorithm uses sequential subsets of the 128 bits from $k$ (this phenomenon is known as slow diffusion rate). Hence, each round of XXTEA relies in a robust function leveraging the immediate previously addressed block. % than the one employed in XTEA. 

%The Corrected Block TEA algorithm (XXTEA)~\cite{wheeler1998correction} keeps the simplicity of XTEA, relying on basic and simple operations. 
%key family features, such as simple implementation, few lines of code and basic operations. 
%XXTEA follows a %This version of the implementation has 
%variable number of rounds depending on the block size to be encrypted. %, e.g., to encrypt 8 bytes, XXTEA performs 12 rounds.
%To address the slow diffusion rate of XTEA, each round of XXTEA follows a robust function, relying on the immediate left neighbor block to process each data block. \abv{o vizinho a esquerda é o bloco predecessor? como a gente nao desenhou o encadeamento de blocos, talvez a gente tenha que ser mais alto nivel} Cryptoanalysis estimates a number of $2^{59}$ simple texts to retrieve the secret key. 

The SKIPJACK algorithm is a 32-round cipher which applies two distinct rules labeled as A and B. These rules are applied interleaved as A, B, A, B per 8 rounds. Permutations comprise of shifts and Feistel's, which use 32 of the 64 bits from the secret key per permutation. %Two approaches adopted for 
Cryptanalyses point out a resistance for attacks of about $2^{48}$, using at least $2^{34}$ plaintexts~\cite{Moh:2015}. % SKIPJACK uses $2^{16}$ bit array.[Não consigo entender esta parte.\rever{este dado é refente a consumo de memoria para realizar o ataque, como o ataque nem sempre é por outro dispositivo limitado talves esta informação seja irrelevante}]}~\cite{biham1999cryptanalysis}.
%
As SKIPJACK, RC2 works on 64-bit blocks and allows a variable key size~\cite{knudsen1998design}. 
It follows two distinct steps: key expansion and encryption. Key expansion can extend any key size, in the range of 1 to 128 bytes, up to a 128-byte key. Encryption performs permutations based on %MIXING and MASHING. {\color{blue}For permutations, RC2 employs %is necessary to use 
a substitution table. %[permutation é o mixing? E como funciona o Mashing?]} 
Estimates to retrieve a secret key are proportional to the effort for analyzing about $2^{34}$ plaintexts~\cite{Moh:2015}.

RC4 is a stream cipher and  
%RC4 employs the same function %can be applied 
%to encrypt and decipher the data flow. % according to the application’s objectives. 
it comprises of %Composed of two parts: the first one presents 
a Key Scheduling Algorithm (KSA) and a Pseudo-Random Generation Algorithm (PRGA). KSA transforms a random key in an initial permutation, whereas PRGA %(Pseudo-Random Generation Algorithm) output, which uses such 
uses this initial permutation to generate a pseudo-random output sequence. Cryptographic transformations applied by the algorithm are linear and simple, using permutations and sums of integer values. %, making it simple and fast. %~\cite{fluhrer2001weaknesses}.
Analyzing RC4 security in a scenario where the protocol allows manipulation of the initial value of the key, the complexity for key recovery is of $2^{13}$ operations% of the proposed attack algorithm
~\cite{orumiehchiha2013cryptanalysis}.

