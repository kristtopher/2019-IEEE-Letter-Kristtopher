%Uma avaliação criteriosa sobre consumo energético e recursos de hardware provocado pela adição de métodos criptográficos objetiva conciliar melhor desempenho e segurança à maior vida útil dos dispositivos vestíveis. Diversos trabalhos sugerem métricas que podem ser aplicadas para analisar o consumo energético em algoritmos criptográficos de cifra leve, tais como~\cite{eisenbarth2012compact,kerckhof2012towards,eisenbarth2007survey}, entre outros. Estes trabalhos apontam métricas relativas ao tamanho do projeto, consumo de memória, números de operações, tempo de relógio e outras. Algumas destas métricas foram consideradas para a análise apresentada neste trabalho. Entretanto, é importante salientar que os trabalhos citados realizam apenas um estudo sobre os softwares. Eles não fazem uma investigação diretamente aplicada em dispositivos vestíveis WBAN.

Adding cryptography methods to wearable devices clearly enhances its security. However, these new features also impact on hardware performance and energy consumption.
A number of works suggest metrics to analyze the energy consumption of lightweight cryptography cipher algorithms, such as as~\cite{eisenbarth2012compact,kerckhof2012towards,eisenbarth2007survey}. These works point out metrics related to the size of the project (\abv{i.e., the hardware dimension}), memory consumption, number of hardware operations and time of clock related to the ciphering task. We consider some of these metrics in the presented work. However, it is important to emphasize that these works focus on evaluating the software implementation of the lightweight cryptography cipher algorithms \abv{chegar isso. voce fala em tamanho do projeto... numero de clocks, consumo de energia} These works do not consider wearable devices or embeded systems.

%Outros trabalhos apresentam uma avaliação geral sobre algoritmos de cifra leve, como~\cite{cazorla2013survey}. É interessante ressaltar que esse trabalho considera a aplicabilidade em microcontrolador MSP430, mesma família empregada nos dispositivos Shimmer utilizados neste artigo. Os autores em~\cite{cazorla2013survey} avaliam o desempenho e consumo de memória com base nos dados do simulador mspdebug. 

In fact, current works just present a general evaluation of lightweight cipher algorithms on embedded systems. For instance, Cazrloa et al.~\cite{cazorla2013survey} consider the applicability of lightweight cipher algorithms using the MSP430 microcontroller which belongs to the same microcontroller family employed in the shimmer devices we use in this work. The performance evaluation has been conducted over simulations, based on mspdebug\abv{footnote pra dizer o que é} data. In this case, authors simulate the microcontroller performance and memory consumption.

%Já outro trabalho~\cite{el2017equalized} apresenta um modelo matemático para consumo de energia para WBANs, tomando como base, três momentos do nó na rede. Sendo eles, o consumo para aquisição de dados, processamento e transmissão. O modelo matemático apresentado considera, o tamanho do pacotes em bits, distância de entre os dispositivos e características do rádio. O modelo ainda foi customizado para atender os modos de baixo consumo \textit{sleep} e \textit{idle}. Entretanto o foco principal desta abordagem é encontrar o consumo equilibrado entre todos os nós da rede. 

The energy consumption evaluation on WBANs is commonly conducted through mathematical models. For example, El Azhari et al.~\cite{el2017equalized} evaluate the energy consumption WBANs on three distinct states: (i) data acquisition, (ii) data processing and, (iii) data transmission. The mathematical model authors present considers the size of packets (in bits), the distance between devices and the characteristics of radio devices. In this work, we customize this model to consider the WBAN modes of operation we evaluate (i.e., sleep and idle modes). \abv{nao entendi a proxima frase} However, the main focus of this approach is find the balanced consumption among all nodes in the network.

%Apesar de não se tratar de um trabalho alusivo à criptografia ou a redes corporais e dispositivos vestíveis, o trabalho~\cite{bessa2017jetsonleap} possui grande importância para a construção deste artigo. Uma vez que apresenta a metodologia a qual nos baseamos para realizar a aferição e coleta dos dados referentes à dissipação de energia pelos dispositivos vestíveis emulando situações reais de uso.

Finally, Bessa et al.~\cite{bessa2017jetsonleap} present a methodology for data acquisition and evaluation of energy consumption on embedded systems. As we previously comment, most of the methodology of the current work relies on Bessa et al.~\cite{bessa2017jetsonleap}, which allows us to model the dissipation of energy by the wearable devices and, as consequence, emulate realistic situations of use.
