%As redes corporais (\textit{Wireless Body Area Networks} - WBANs) são formadas por uma gama de dispositivos computacionais especiais, dispostos pelo corpo. Estes dispositivos atuam de modo cooperativo facilitando consideravelmente a medição contínua de parâmetros fisiológicos do corpo humano, sua principal aplicação. Consequentemente, é possível  tornar o cuidado com a saúde algo viável em qualquer lugar e a qualquer momento. Por se tratarem de dispositivos especiais, com grande limitações de recursos de hardwares, assegurar a segurança, confiabilidade e disponibilidade dos dados significa proteger vidas humanas~\cite{ali2013energy}. Uma vez que os dados coletados e propagados pelos sensores vestíveis envolvem dados pessoais sigilosos.

The wireless body area networks (WBANs) comprise a range of wearable devices, e.g., wearable sensors, smart watch, etc, arranged on the body. These devices operate cooperatively making easier the the continuous monitoring of physiological parameters of the human body. Consequently, it allows healthcare anywhere, anytime. Because these are special devices, with great limitations on hardware, ensuring security, confidentiality and availability means to protect data and human lives~\cite{ali2013energy}, since collected data and its transmission by wearable sensors involve sensitive personal data.

%Os sensores vestíveis representam um dos tipos de dispositivos que formam as redes corporais. Como o próprio nome indica, estes são dispostos em torno do corpo. Seu arranjo busca monitorar continuamente os sinais vitais e o comportamento dos usuários sem interferir na rotina destes. As pesquisas nesta área têm ganho uma importância crescente, principalmente pelos avanços tecnológicos na biomedicina e na miniaturização dos sensores~\cite{bandodkar2014non}. Em contrapartida, os pesquisadores ainda se deparam com grandes desafios. Entre eles a autonomia energética, dado que a substituição ou recarga de baterias tendem a ser um trabalho complicado. Outro grande desafio é a segurança na transmissão dos dados coletados, uma vez que os dispositivos devem garantir a confidencialidade, integridade e disponibilidade dos dados. Estas tarefas devem ser implantadas de modo que não comprometam a eficiência energética e os recursos de hardware dos sensores vestíveis~\cite{fletcher2010wearable}.

Wearable sensors are one of the types of devices forming WBANs. %As the name implies, they are arranged around the body.
They search for continuously monitoring vital signs and user's  behavior, without interfering with their routine. Research in this area has gained increasing importance, particularly due to technological advances in biomedicine and sensor iniaturization~\cite{bandodkar2014non}. In contrast, researchers still face major challenges, such as the energy consumption, once the replacement or battery recharging tend to be a complex task. Another major challenge is security in data transmission, since the devices must ensure data confidentiality, integrity and availability.

%Este artigo avalia métodos de segurança que se alinham com os requisitos das redes sensores corporais. Portanto, seu objetivo é apresentar o melhor \textit{trade-off} entre segurança e consumo de recursos, principalmente o energético. Para isto, a análise foca em métodos criptográficos conhecidos por utilizarem chave simétrica, ou seja, métodos em que os dispositivos comunicantes conhecem, à priori, o segredo (chave) utilizado para criptografar as mensagens enviadas~\cite{mandal2012evaluation}. A qualidade da segurança tem relação direta com o tamanho da chave, ou seja, quanto maior mais eficaz~~\cite{jorstad1997cryptographic}~\cite{mandal2012evaluation}. Entretanto, a criptografia de chave simétrica emprega uma chave menor em comparação à criptografia de chave pública oferecendo segurança equivalente. Além de consumir menos energia e memória~\cite{liu2010hybrid,mandal2012evaluation}.

This letter presents results of an empirical evaluation of security methods aligned with the requirements of WBANs. Hence, its goal lies to present the best tradeoff between security and resource consumption, particularly energy. For this, our analysis focuses on symmetric cryptographic methods, in which the communicating devices know a priori the key (secret) employed to encrypt messages\cite{mandal2012evaluation}. The security strength is directly related to the key size, i.e., the larger, more secure~\cite{jorstad1997cryptographic,mandal2012evaluation}. However, the symmetric key cryptography employs a short key compared to public key cryptography, offering equivalent safety and consuming less power and memory~\cite{liu2010hybrid,mandal2012evaluation}.

%Os métodos avaliados dispõem de características de projetos que permitem lidar com os limiares entre segurança, custo e desempenho. Estes são conhecidos pelo termo genérico ``criptografia leve''. Particularmente, este artigo avalia os métodos XTEA, XXTEA, SKIPJACK e RC2. Todos pertencem à classe de cifra de bloco, onde as operações de cifra são realizadas sobre um bloco de dados. Diferente destes o artigo também foca em método de cifra de fluxo, o RC4, em que trabalha sobre um fluxo de bits de tamanho conhecido. Deste modo contemplamos métodos com propriedade distintas. Outros métodos ainda foram considerados, tais como, KSEED, TWOFISH e CAST5. Entretanto, revelaram-se inviáveis, uma vez que consomem demasiadamente os recursos de hardware.

All methods have features that allow them to handle thresholds between security, cost and performance. These is why they are known as "light cryptography" methods. Particularly, this letter assesses the XTEA, XXTEA, SKIPJACK and RC2 methods. They all belong to the class of block cipher where the encryption operations are performed on a data block. This letter also focuses on a flow encryption method, RC4, which works on flow of bits of known size. Thus, we analyze methods with different properties. Also, other methods have been considered to this study, such as KSEED, TWOFISH and CAST5. However, they proved to be unfeasible to wearable devices, since they consume a high amount of hardware resources.

%Deste modo, implementamos uma biblioteca criptográfica para ser aplicada em dispositivos (nós) vestíveis sem fio produzidos pela empresa Shimmer~\cite{burns2010shimmer}, modelo 2R. Para classificar e mensurar a eficiência energética correspondente a cada método utilizamos um circuito contendo um aquisitor de dados de baixo custo, um resistor e um computador, adaptando a metodologia~\cite{bessa2017jetsonleap}. Deste modo, avaliamos a dissipação real de energia para os principais cenários cabíveis de um nó sensor na rede, \textit{sleep}, \textit{idle} e \textit{run}. Este último, ainda fora examinado utilizado os padrões de comunicação IEEE 802.15.1 (Bluetooth) e 802.15.4 (ZigBee), ambos comumente aplicados em redes corporais. Julgamos ainda a correlação entre a quantidade de operações lógicas e aritméticas necessárias para cifrar um bloco de dados por cada método junto ao seu respectivo consumo energético. 

Thus, we have implemented a cryptography library to be applied in wireless wearable devices (nodes) produced by the Shimmer company~\cite{burns2010shimmer}, 2R model. For classification and measurement of energy consumption, for each method we use a circuit containing a low cost aquisitor, a resistor and a computer, adapting the methodology in~\cite{bessa2017jetsonleap}. Thus, we evaluated the actual power dissipation for the main scenarios applicable to a sensor node, as sleep, idle and run. The run scenario has also been examined following the two main communication standards: IEEE 802.15.1 (Bluetooth) and 802.15.4 (ZigBee). We also believe that the correlation between the amount of arithmetic and logic operations required to encrypt a data block for each method goes along with their respective energy consumption. Based on the experiments and result analysis, we were able to verify the use of these cryptographic methods in WBANs. We highlight the SKIPJACK method as the most efficient than among others. As well as we point the advantage of the ZigBee technology compared to Bluetooth, as regards the dissipation of energy.

%\textbf{resultados}
%Tomando como base os experimentos realizados e a análise análise de seus resultados, fomos capazes de ratificar o emprego destes métodos criptográficos em redes corporais. Destacando o método SKIPJACK como o mais eficientes entre os demais avaliados. Assim como a vantagem da tecnologia ZigBee em relação ao Bluetooth, no que se refere a dissipação de energia.


%O restante deste artigo está organizado como segue. A Seção~\ref{sec:Related_Work} discute sobre os trabalhos relacionados. Na Seção~\ref{sec:Background}, detalhamos os conceitos referentes aos sensores vestíveis e os protocolos de comunicação utilizados, além de expor os métodos criptográficos abordados. A Seção~\ref{sec:Methodology} descreve a metodologia utilizada para aferir o consumo energético, além de abordar outras métricas. Na Seção~\ref{sec:Results}, são apresentados os resultados obtidos. E a Seção~\ref{sec:Conclusion} expõe as considerações finais sobre o artigo e direciona os trabalhos futuros.

The letter proceeds as follows. Section~\ref{sec:Related_Work} discusses related works. Section\ref{sec:Background} details the concepts relating to wearable sensors and communication protocols. Section~\ref{sec:Methodology} describes the evaluation methodology. Section~\ref{sec:Results} presents results. And Section~\ref{sec:Conclusion} presents the final consideration of the letter and future works.