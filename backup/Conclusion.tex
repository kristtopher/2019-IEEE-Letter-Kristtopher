%Após os experimentos e análise dos resultados referentes às métricas abordadas é possível concluir que os métodos de criptografia de cifra de bloco proporcionam uma melhor eficiência em relação à segurança e energia comparado a cifra de fluxo do RC4. Ainda podemos destacar o melhor desempenho geral para o método Skipjack, o credenciando como melhor técnica a ser implementada em dispositivos de sensoriamento com recursos escassos, tais como as WBANs. Isto devido ao fato de apresentar um menor consumo médio em relação aos demais. O Skipjack conta ainda com o segundo menor consumo de recursos de memória, além de uma complexidade de recuperação de chaves considerável para algoritmos de criptografia de chave simétrica. Consideramos ainda que o \textit{overhead} para cifrar os dados é desprezível perante a segurança promovida e a dissipação da energia ocasionada pelos módulos de transmissão via rádio. Como sequência deste trabalho propomos estender nossa biblioteca criptográfica para dispositivos vestíveis com métodos referentes à criptografia de curva elíptica e modelos híbridos. Uma vez que, estas abordagem se comprometem em prover maior segurança com razoável aumento de consumo de recursos de hardwares~\cite{liu2010hybrid}. Deste modo, podemos aferir e comparar o consumo  energético e de recursos computacionais entre as distintas classes e métodos de criptográficos.

%\rever{
In this letter, we investigate the impact of block and stream ciphers in security, resource usage and power consumption for wearable devices. Differently from the state-of-the-art, we performed hardware-driven power consumption measurements under two communication standards from the IEEE 802.15 family. The SKIPJACK algorithm presented the best performance for wearable devices with constrained resources once it presented the lowest average energy consumption among the other evaluated algorithms and the second least memory resource usage. Moreover, this algorithm offers a considerable key recovery complexity for symmetric key cryptography algorithms. Our results also pointed out that the overhead to data encryption is negligible when compared to the offered security and low energy dissipation level for all implemented and evaluated algorithms. %caused by the radio transmission modules.}

%\rever{After the experiments and analysis of the results regarding the metrics addressed, it is possible to conclude that the block cipher cryptography methods provide better efficiency in relation to the safety and energy compared to the RC4 flow cipher. We can still highlight the best general performance for the Skipjack method, which is in this case, the best technique to be implemented in wearable devices with scarce resources. This is due to the fact that it presents a lower average consumption in relation to the others. Skipjack also has the second least memory resource consumption, as well as a considerable key recovery complexity for symmetric key cryptography algorithms. We also consider that the overhead to encrypt the data is negligible in view of the security promoted and the dissipation of the energy caused by the radio transmission modules. As a result to this paper, we propose to extend our cryptographic library to wearable devices with methods related to elliptic curve cryptography and hybrid models, since these approach compromise to provide greater security with a reasonable increase of consumption of hardware resources. Hence, we can measure and compare the energy and computational resources consumption among the different cryptographic classes and methods.}\analisar{[precisamos revisar ainda]}